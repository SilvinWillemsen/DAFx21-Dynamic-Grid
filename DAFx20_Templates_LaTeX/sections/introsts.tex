\section{Introduction}

Finite-difference time-domain (FDTD) methods are flexible and generalisable techniques which have recently gained an increased interest in the physical modelling of musical instruments [CITE]. %\SBcomment[OK, I think reviewers may not like this at all. Particularly in the case of the trombone, where people like Perry Cook were doing things with variable delay lines (I think) a long time ago. if a waveguide person gets this for review, they will probably get annoyed. ] 
FDTD methods subdivide continuous partial differential equations (PDEs) that describe the physics of the instrument at hand into a grid of discrete points in space and time. 
There is, however, an inherent challenge when trying to implement systems with dynamic parameters' variations using FDTD methods: the number of grid points used in the simulation is closely tied to the parameters themselves through a stability condition.
Adding and removing points from the grid is therefore nontrivial and could cause audible artefacts or instability of the simulation. Full-grid interpolation \cite[Ch. 5]{bilbao2009} could be a possible solution here, but extremely high sample rates are necessary to avoid audible artefacts and low-passing effects when changing between grid configurations. 

This article proposes an efficient method to smoothly add and remove points from 1D finite-difference grids to allow for dynamic parameter changes. We are interested in `slowly' varying parameters (sub-audio rate) \SWcomment[so that modal and energetic analysis techniques are still meaningful to a degree].

Any musical instrument can be subdivided into an exciter and a resonator component \cite{Borin1989}. \SWcomment[$\leftarrow$ watch out here.. there is also an interface between the two] Examples of exciter-resonator combinations are the bow and violin and the lips and trumpet. In nearly all instruments, the parameters describing the exciter are continuously varied by the performer to play the instrument. As an example, the bow-velocity, position and pressure for the violin, and lip pressure and frequency for the trumpet. Naturally, the resonator is altered by fingering the strings of the violin or pressing valves on the trumpet to change their pitches. In the real world, however, physical dimensions of the resonators do not change: the string-length stays the same and the total tube length remains unchanged, only the parts that resonate are shortened or lengthened.

While the physical dimensions of the instrument do not change, there are several examples where the parameters of the resonator are also modified. A prime example of this is the trombone, where the tube length is dynamically changed in order to generate different pitches. The slide whistle is another example in this category. Guitar strings are another category where the tension can be smoothly modulated during performance using the fretting finger, a whammy bar or even the tuning pegs directly (see \cite{Gomm2011}) to create smooth pitch glides. The same kind of tension modulation is used for the membranes of timpani or ``hourglass drums" to change the pitch. %or pressing down on the strings of a hammered dulcimer on one side of a bridge, while playing the string on the other side for the same effect \cite{Glenn2014}. 
It is these direct parameter modifications of the resonators that we are interested in to simulate.
%Among the reasons of why one would simulate an instrument rather than sample an existing one, is that one could extend the capabilities of this instrument beyond what is physically possible, such as changing material properties or size of the instrument dynamically.

% of simulating musical instruments rather than recording them and playing them back (samples), is the flexibility of control and playability of the instrument. Imagine trying to record the entire control space of a violin, i.e., all possible combinations of bowing force, bowing velocity and fingering positions. The recording procedure would take a great amount of time and resources, let alone the amount of data storage required to store all of the high-quality audio.

% Due to the vast parameter space of many instruments, using simulations is a much more viable solution for capturing the full expressivity of an instrument than sampling.
% Some instruments require their parameters to be changed in order to be played. The first instrument that comes to mind is the trombone, where the geometry of the system, i.e., the tube length, is dynamically varied to produce different pitches. 

Other than simulating existing instruments, one can simulate instruments that can be manipulated in physically impossible ways. %\SBcomment[OK, it's good to have this in here, but this is immediately kind of a red flag for JASA, where they'd expect you to be attacking a problem from the real world first. Otherwise, they might well say that this belongs in CMJ. So: suggesting referring to real world examples first, then bring up the idea of imaginary instruments... ] \SWcomment[Will do!] 
Examples of this could be to dynamically change material properties such as density or stiffness, or even the geometry and size of the instrument where this is physically impossible. %An example of a real-world instrument that requires these manipulations is the trombone, where tube length is dynamically controlled by the player.

% This can be extended from musical instruments to rooms and acoustic simulation with variable shapes

 

\SWcomment[Only if there is space $\rightarrow$] This paper is structured as follows:
