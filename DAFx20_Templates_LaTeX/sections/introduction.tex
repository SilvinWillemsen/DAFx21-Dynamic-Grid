\section{Introduction}
\SWcomment[TODO: Decide on modelling or modeling]

Simulating musical instruments using physical modelling is a well-established field. Among the reasons of why one would simulate an instrument rather than sample an existing one, is that one could extend the capabilities of this instrument beyond what is physically possible, such as changing material properties or size of the instrument dynamically. \SBcomment["on the fly" is too informal. Maybe "under dynamic playing conditions"].

\SWcomment[Much more intro here obviously]{}
% of simulating musical instruments rather than recording them and playing them back (samples), is the flexibility of control and playability of the instrument. Imagine trying to record the entire control space of a violin, i.e., all possible combinations of bowing force, bowing velocity and fingering positions. The recording procedure would take a great amount of time and resources, let alone the amount of data storage required to store all of the high-quality audio.

% Due to the vast parameter space of many instruments, using simulations is a much more viable solution for capturing the full expressivity of an instrument than sampling.

One of the incentives of simulating the physics of musical instruments rather than sampling their real-world counterparts, is that the virtual instrument can be manipulated in physically impossible ways. \SBcomment[OK, it's good to have this in here, but this is immediately kind of a red flag for JASA, where they'd expect you to be attacking a problem from the real world first. Otherwise, they might well say that this belongs in CMJ. So: suggesting referring to real world examples first, then bring up the idea of imaginary instruments... ] \SWcomment[Will do!] Examples of this could be to change material properties, or even the shape of the instrument on the fly\SBcomment[See comment in abstract]. An example of a real-world instrument that requires these manipulations is the trombone, where tube length is dynamically controlled by the player.

% This can be extended from musical instruments to rooms and acoustic simulation with variable shapes

Existing physical modelling techniques include mass-spring systems [REF] and digital waveguides [REF]. \SWcomment[SOTA on trombone / variable parameters with waveguides here!]



Modal synthesis [REF], though requiring some assumptions and simplifications for most systems, does allow for easy and smooth parameter changes -- as seen in [REF] and [REF] -- and could thus be a good candidate for implementing the aforementioned manipulations. In the case of the trombone, and due to its non-homogenous geometry, there is no closed-form solution available and the modes would need to be calculated for every single slide configuration. Finite-difference time domain (FDTD) methods on the other hand, though generally more computationally expensive than other techniques, are more flexible and generalisable and do not need as many simplifications as modal synthesis does.\SBcomment[OK, I think reviewers may not like this at all. Particularly in the case of the trombone, where people like Perry Cook were doing things with variable delay lines (I think) a long time ago. if a waveguide person gets this for review, they will probably get annoyed. ] These methods subdivide continuous partial differential equations (PDEs) that describe the physics of the system at hand into a grid of discrete points in space and time. 
%
% Below, a brief introduction about grids and their relationship to stability is explained.
%
% \subsection{Grids and stability}
The distance between two discrete points in space (the grid spacing) and the time step between two discrete moments in time are closely connected through a \textit{stability condition}. This condition dictates the maximum number of points allowed to describe system before it gets unstable and the simulation ``explodes''. The closer the grid spacing is to the stability condition, the higher the quality of the simulation. If the condition is satisfied with equality, the quality of the simulation is at a maximum. \SBcomment[OK, this whole argument above seems out of place at little. Plus there are no references, or an indication of what quality means. As in: reviewers probably will not grasp why this is an important problem to solve. I suggest removing. Better to focus attention here away from the detailed numerical details (like grid spacings and numerical stability conditions and stuff) as the reviewers will probably see these as just small technical details. The important thing is the dynamic behaviour...] \SWcomment[Yes, I completely agree, the introduction mainly consists of notes at this point, some that were left from when I started writing this paper long ago :)]
Furthermore, the stability condition depends on the parameters of the model, such as material properties or size of the system, i.e., parameters that could be changed on the fly \SBcomment[As above].

This article proposes a method to smoothly add and subtract points from a system in real time so that the stability condition is always satisfied with equality. This allows for a simulation where the parameters can be changed dynamically without running into either stability or quality issues. One can consider this work an introduction to implementing the trombone using FDTD methods.

This article is structured as follows 
\begin{itemize}
% \item Section 2: Background on PDE $\rightarrow$ FDTD $\rightarrow$ stability conditions 
\item Section 2: Continuous systems
\item Section 3: Numerical methods
\item Section 4: Dynamic grid
\item Section 5: Results
\item Section 6: Discussion
\item Section 7: Conclusions and Future work
\end{itemize}

