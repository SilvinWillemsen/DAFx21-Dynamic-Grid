\section{Introduction}

\SWcomment[something before this] Any musical instrument can be subdivided into an exciter and a resonator component \cite{Borin1989}. Examples of exciter-resonator combinations are the bow and violin and the lips and trumpet. In nearly all instruments, the variables describing the exciter are `modified' to play the instrument, i.e, the bow-velocity, position and pressure for the violin, and lip pressure and frequency for the trumpet. Naturally, the resonator is altered by fingering the strings of the violin or pressing valves on the trumpet to change its pitch. The variables describing the physics of the resonator itself are not changed though; the string-length stays the same and the total tube length remains unchanged, only the parts that resonate are shortened or lengthened.

In some cases, though, the parameters describing the resonator are modified. A prime example of this is the trombone, where the tube length is dynamically changed in order to generate different pitches. The slide whistle is another example in this category. Guitar strings are another category where the tension can be smoothly modulated during performance using the fretting finger, a whammy bar or even the tuning pegs directly (see \cite{Gomm2011}) to create smooth pitch glides. The same kind of tension modulation is used for the membranes of timpani or ``hourglass drums" to change the pitch. %or pressing down on the strings of a hammered dulcimer on one side of a bridge, while playing the string on the other side for the same effect \cite{Glenn2014}. 
It is these direct parameter modifications of the resonators that we are interested in to simulate.
%Among the reasons of why one would simulate an instrument rather than sample an existing one, is that one could extend the capabilities of this instrument beyond what is physically possible, such as changing material properties or size of the instrument dynamically.

% of simulating musical instruments rather than recording them and playing them back (samples), is the flexibility of control and playability of the instrument. Imagine trying to record the entire control space of a violin, i.e., all possible combinations of bowing force, bowing velocity and fingering positions. The recording procedure would take a great amount of time and resources, let alone the amount of data storage required to store all of the high-quality audio.

% Due to the vast parameter space of many instruments, using simulations is a much more viable solution for capturing the full expressivity of an instrument than sampling.
% Some instruments require their parameters to be changed in order to be played. The first instrument that comes to mind is the trombone, where the geometry of the system, i.e., the tube length, is dynamically varied to produce different pitches. 

Other than simulating existing instruments, one can simulate instruments that can be manipulated in physically impossible ways. %\SBcomment[OK, it's good to have this in here, but this is immediately kind of a red flag for JASA, where they'd expect you to be attacking a problem from the real world first. Otherwise, they might well say that this belongs in CMJ. So: suggesting referring to real world examples first, then bring up the idea of imaginary instruments... ] \SWcomment[Will do!] 
Examples of this could be to dynamically change material properties such as density or stiffness, or even the geometry and size of the instrument. %An example of a real-world instrument that requires these manipulations is the trombone, where tube length is dynamically controlled by the player.

% This can be extended from musical instruments to rooms and acoustic simulation with variable shapes

Simulating musical instruments using physical modelling is a well-established field. Existing physical modelling techniques that have shown their ability to implement dynamic parameter changes of resonators include digital waveguides (DWGs) \cite{Smith1992} and modal synthesis \cite{morrison1993mosaic}. Literature using DWGs for time-varying parameters include Michon's BladeAxe \cite{Michon2014,Michon2016} and Cook's ... \SWcomment[SOTA on trombone / variable parameters with waveguides here!]

Modal synthesis, though requiring some assumptions and simplifications for most systems, does allow for easy and smooth parameter changes. Resonant frequencies of the different modes can be dynamically modified without many issues as long as modal frequencies exceeding the stability condition are excluded as parameters are changed. Examples can be found in \cite{Willemsen2017} and \cite{Mehes2016, Mehes2017, Walstijn2017} \SWcomment[$\leftarrow$ maybe not all of these van Walstijn papers]. Modal synthesis could thus be a good candidate for implementing the aforementioned dynamic parameters. In the case of the trombone, however, due to its non-homogenous geometry, there is no closed-form solution available and the modes would need to be calculated for every single slide configuration.

Finite-difference time-domain (FDTD) methods on the other hand, though generally more computationally expensive than other techniques, are flexible and generalisable and do not need as many simplifications as modal synthesis does. %\SBcomment[OK, I think reviewers may not like this at all. Particularly in the case of the trombone, where people like Perry Cook were doing things with variable delay lines (I think) a long time ago. if a waveguide person gets this for review, they will probably get annoyed. ] 
FDTD methods subdivide continuous partial differential equations (PDEs) that describe the physics of the system at hand into a grid of discrete points in space and time. \SWcomment[slightly different wording here $\rightarrow$] The challenge when trying to implement systems with dynamic parameters using FDTD methods, is that the number of grid points used in the simulation is closely tied to the parameters themselves \SWcomment[through a stability condition]. Adding and removing points from the grid is nontrivial and could cause audible artefacts or instability of the simulation. Full-grid interpolation \cite[Ch. 5]{bilbao2009} could be a possible solution here, but extremely high sample rates are necessary to avoid audible artefacts and low-passing effects when changing between grid configurations. 

% Below, a brief introduction about grids and their relationship to stability is explained.
%
% \subsection{Grids and stability}
% The distance between two discrete points in space (known as grid spacing) and the time step between two discrete moments in time are closely connected through a \textit{stability condition}. This condition dictates the maximum number of points allowed to describe a system before it gets unstable and the simulation ``explodes''. The closer the grid spacing is to the stability condition, the higher the quality of the simulation. If the condition is satisfied with equality, the quality of the simulation is at a maximum. \SBcomment[OK, this whole argument above seems out of place at little. Plus there are no references, or an indication of what quality means. As in: reviewers probably will not grasp why this is an important problem to solve. I suggest removing. Better to focus attention here away from the detailed numerical details (like grid spacings and numerical stability conditions and stuff) as the reviewers will probably see these as just small technical details. The important thing is the dynamic behaviour...] \SWcomment[Yes, I completely agree, the introduction mainly consists of notes at this point, some that were left from when I started writing this paper long ago :)]
% Furthermore, the stability condition depends on the parameters of the model, such as material properties or size of the system, i.e., parameters that could be changed on the fly \SBcomment[As above].

This article proposes an efficient method to smoothly add and remove points from 1D finite-difference grids to allow for dynamic parameter changes. We are interested in `slowly' varying parameters (sub-audio rate) \SWcomment[so that modal and energetic analysis techniques are still meaningful to a degree]. One can consider this work an introduction to implementing the trombone using FDTD methods.

\SWcomment[Only if there is space $\rightarrow$] This paper is structured as follows:
