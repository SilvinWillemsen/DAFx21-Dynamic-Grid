\section{Continuous Systems}\label{sec:continuous}
%
%going from a partial differential equation (PDE) to an update equation that can be implemented.
%
%using the famous 1D wave equation.
%
% The physics of dynamic systems is commonly described using partial differential equations (PDEs) operating in continuous time. To aid the illustration of the proposed method, the 1D wave equation will be used. This does not mean that the method is limited to this, and could be extended to other systems, possibly even higher dimensional ones.

The wave equation is a useful starting point for investigations of time-varying behaviour in musical instruments. In 1D, the wave equation may be written as
\begin{equation}\label{eq:1dwave}
    \frac{\partial^2 \ugen}{\partial t^2}= c^2\frac{\partial^2 \ugen}{\partial x^2}\ ,
\end{equation}
and is defined over spatial domain $x \in [0, L]$, for length $L$ (in m) and time $t \geq 0$ (in s). $c$ (in m/s) is the wave speed. The dependent variable $\ugen = \ugen(x,t)$ in Eq. \eqref{eq:1dwave} may be interpreted as the transverse displacement of an ideal string, or the acoustic pressure in the case of a cylindrical tube. Two possible choices of boundary conditions are
\begin{subequations}\label{eq:continuousBoundaries}
    \begin{align}
        \ugen(0, t) = \ugen(L, t) &= 0\quad \text{(Dirichlet)},\label{eq:contDirichlet}\\
        \frac{\partial}{\partial x} \ugen(0, t) = \frac{\partial}{\partial x} \ugen(L, t) &= 0\quad \text{(Neumann)},\label{eq:contNeumann}
    \end{align}
\end{subequations}
and describe `fixed' or `free' boundary respectively in the case of an ideal string, and `open' or `closed' conditions respectively in the case of a cylindrical acoustic tube.

\subsection{Dynamic parameters}\label{sec:dynamicParamsCont}
In the case of the 1D wave equation, only the wave speed $c$ and length $L$ can be altered (in the case of an acoustic tube, only $L$ is variable, and for a string, $c$ could exhibit variations through changes in tension). If the same boundary condition is used at both ends of the domain, and under static conditions, the fundamental frequency $f_0$ of vibration can be calculated according to
\begin{equation}\label{eq:fundamentalFreqCont}
    f_0 = \frac{c}{2L}\ .
\end{equation}
In the dynamic case, and under slow (sub-audio rate) variations of $c$ or $L$, Eq. \eqref{eq:fundamentalFreqCont} still holds approximately.
% \SBcomment[OK, but this notion of a frequency is only true under static conditions...kind of confusing] \SWcomment[Is it? It should be true for the dynamic case as well right?]\SBcomment[Well, I guess under slow variations, you can sort of say that there are still modes...but in general once you have time varying behaviour, you lose the ability to do Fourier analysis easily, and the modal analysis is not strictly correct] \SWcomment[Hmm.. but if this is true for any static combination of $c$ and $L$ (which it is right?) wouldn't it be true for the dynamic case as well?]\SBcomment[No way. It's approximately true for sub-audio rate variation in $c$. But once the variation gets higher, you will start to see AM effects (sidebands!).All modal analysis requires LTI behaviour. You can say that if the rate of variation of $c$ is sufficiently slow, then the modal frequencies above are approximately valid... ] \SWcomment[Ah of course, gotcha!]
%
From Eq. \eqref{eq:fundamentalFreqCont}, one can easily conclude that in terms of fundamental frequency, halving the length in Eq. \eqref{eq:1dwave} is identical to doubling the wave speed and vice versa. Looking at Eq. \eqref{eq:1dwave} in isolation, $f_0$ is the only behaviour that can be changed. One can thus leave $L$ fixed %($L = 1$ for simplicity) 
and allow time variation in $c$, so that $c = c(t)$, which will prove easier to work with in the following sections. This fact can more easily be seen if Eq. \eqref{eq:1dwave} is scaled or non-dimensionalised as in \cite{bilbao2009}, where scaled domain  $x' = x/L\  \Rightarrow \ x'\in[0, 1]$ and $\gamma = c/L$ such that $f_0 = \gamma / 2$. For clarity, however, we will employ a fully dimensional representation here.
% \SBcomment[OK: here, need to refer back to the two primary cases here: the trombone, and the string under variable tensioning, and justify the use of a time-varying $c$ only in these cases. Otherwise this becomes too abstract. ] \SWcomment[I feel like I need to talk about scaling here and how this confirms this statement ($f_0 = \gamma/2$ with $\gamma = c/L$ and scaled space $x \in {[}0, 1{]}$).] \SBcomment[Well, it looks like in the rest of the paper, you've got everything dimensional. Do you want to go for a fully non-dimensional domain $x\in(0,1)$, so you have $\gamma$ in there? Then you could deal with this issue right here. ] \SWcomment[hmm.. isn't there something we can say about the 1D wave and how this applies here? I want to keep this general (unscaled).. also as I am going to change $L$ in the trombone paper.. something like what I wrote now? ]
% \SWcomment[I changed it to this what do you think?]