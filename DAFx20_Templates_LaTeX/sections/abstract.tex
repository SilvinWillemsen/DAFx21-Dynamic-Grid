Many techniques of modelling musical instruments exist of which time-stepping methods (e.g., finite-difference time-domain (FDTD) methods) are considered the most flexible and generalisable in terms of the type of systems they can model, both linear and nonlinear. These methods do, however, lack the capability of handling smooth parameter changes while retaining optimal simulation quality, something other techniques are better suited for. \SBcomment[Not sure this is completely true either...the reviewers will definitely point out examples from the massive FDTD literature (like CFD e.g., where they use immersed boundary techniques for fluid structure interaction with moving boundaries). ]This article proposes a method to dynamically alter the grids of simulations based on FDTD methods by smoothly adding and removing grid points from the system. This allows for dynamic parameter changes in physical models of musical instruments which are based on this technique. An instrument such as the trombone can now be modelled using FDTD methods, as well as physically impossible instruments. %Furthermore, this technique allows the stability condition that the schemes using FDTD methods are based on, to always be satisfied with equality and thus have the highest simulation quality possible.\SBcomment[Too technical for the abstract: can say "Stability analysis follows directly as in the static case"] \SWcomment[Does it? I haven't tried to perform stability analysis on the method yet..]\SBcomment[I don't know!] 