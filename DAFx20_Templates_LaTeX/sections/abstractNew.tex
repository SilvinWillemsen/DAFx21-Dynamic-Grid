Many techniques to physically model musical instruments exist, of which time-stepping methods (e.g., finite-difference time-domain (FDTD) methods) are considered the most flexible and generalisable in terms of the type of systems they can model. These methods do, however, lack the capability of handling smooth parameter changes %-- such as necessary for modelling the trombone -- 
while retaining optimal simulation quality and stability, something other techniques are better suited for. \SWcomment[Will reviewers comment on this? You talked about immersed boundary conditions in CFD literature, but it is beyond my abilities to make any statements on that]  In this paper, we propose an efficient method to smoothly add and subtract grid points from a FDTD system under sub-audio rate parameter variations. This allows for dynamic parameter changes in physical models of musical instruments which are based on this technique. An instrument such as the trombone can now be modelled using FDTD methods, as well as physically impossible instruments where fx. material density or its geometry can be made time-varying. Results show that the method does not produce audible artefacts though further investigation regarding stability still needs to be done.

% An instrument such as the trombone can now be modelled using FDTD methods, as well as physically impossible instruments. %Furthermore, this technique allows the stability condition that the schemes using FDTD methods are based on, to always be satisfied with equality and thus have the highest simulation quality possible.\SBcomment[Too technical for the abstract: can say "Stability analysis follows directly as in the static case"] \SWcomment[Does it? I haven't tried to perform stability analysis on the method yet..]\SBcomment[I don't know!] 
