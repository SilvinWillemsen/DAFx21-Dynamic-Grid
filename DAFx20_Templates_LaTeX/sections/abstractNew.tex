For physical modelling sound synthesis, many techniques are available; time-stepping methods (e.g., finite-difference time-domain (FDTD) methods) have an advantage of flexibility and generality in terms of the type of systems they can model. These methods do, however, lack the capability of easily handling smooth parameter changes 
while retaining optimal simulation quality and stability, something other techniques are better suited for. In this paper, we propose an efficient method to smoothly add and remove grid points from a FDTD simulation under sub-audio rate parameter variations. This allows for dynamic parameter changes in physical models of musical instruments. An instrument such as the trombone can now be modelled using FDTD methods, as well as physically impossible instruments where parameters such as e.g. material density or its geometry can be made time-varying. Results show that the method does not produce audible artefacts and stability analysis is ongoing.

% An instrument such as the trombone can now be modelled using FDTD methods, as well as physically impossible instruments. %Furthermore, this technique allows the stability condition that the schemes using FDTD methods are based on, to always be satisfied with equality and thus have the highest simulation quality possible.\SBcomment[Too technical for the abstract: can say "Stability analysis follows directly as in the static case"] \SWcomment[Does it? I haven't tried to perform stability analysis on the method yet..]\SBcomment[I don't know!] 
