\section{Discussion}
To decide whether the proposed method works satisfactory, the results presented in the previous section are compared to the method requirements listed in Section \eqref{sec:methodReq}. 



As the error of $f_1$ is  first requirements The fact that he highest mode is not excited is probably due to the restriction imposed on the two moving but connected points through the rigid connection.

% The fact that identical behaviour is observed at integer $N$ proves that at integer $N$ the method indeed reduces to the normal case confirming Section \ref{sec:systSetup}. If the system is excited when $N$ is not an integer, the highest mode will also be excited.


All issues with the proposed method happen in higher frequencies. Physical processes usually lose the highest frequencies first

Due to high-frequency losses in physical systems. 

Higher modes

As can be seen from the figure, the lower harmonic partials of both systems line up almost exactly before the partials interpolated start to deviate towards the lower end of the spectrum.

($<0.2\%$ of their respective frequency, $<1.3\%$ of the fundamental $f_0$). 


Applications could be non-linear systems where parameters are modulated based on the state of the system. 

The proposed method does not provide the exact solution to the problem, but does circumvent the need for upsampling and higher orders of computations necessary to approximate this solution. Even though interpolation needs to happen, the drawbacks of full-grid interpolation can be avoided by not `listening' to the location where points are added but rather closer to the boundary. If one wants to listen to the center, the location where points are added or removed can easily be changed.


frequency domain, the locations of the partials comparing the discrete 1D wave with $N = 30.5$ and $f_\text{s} = 44100$ (interpolation needs to happen) with $N = 61$ and $f$ 

If the amount that a parameter changes within a small enough period of time (give values here, hopefully referring back to the results) the fact that points are added at a specific location (rather than distributed over the grid) will not matter as... 

The 


Note: Reference in Stefan's book about frequency sensitivity > 3000 Hz: E. Zwicker and H. Fastl. Psychoacoustics: Facts and Models. Springer-Verlag, Berlin-Heidelberg,
Germany, 1990.
