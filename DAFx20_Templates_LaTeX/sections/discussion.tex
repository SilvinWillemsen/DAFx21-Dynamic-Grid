\section{Discussion}
To decide whether the proposed method works satisfactory, the results presented in the previous section are compared to the method requirements listed in Section \ref{sec:methodReq} (denoted by r\# for short). 

It can be argued that the frequency deviations of $f_1$ from $f_0$ are sufficiently small to say that the r1 is satisfied. As for r2, a fractional number of intervals $\Nfrac$ has been introduced and smooth transitions are indeed observed from Figure \ref{fig:spectrogram}, in the case when $c$ is decreased and $\Nfrac$ is increased. When $c$ is increased instead, transitions are not smooth as artefacts are generated when grid points are removed. This problem is solved by the displacement correction presented in Section \ref{sec:dispCorr}. However, the filtering effect that the displacement correction has on the system (mentioned in Section \ref{sec:dispCorrRes}) is not ideal as it creates notches in the spectrum of the output sound, especially for a (relatively) static system. The least intrusive filtering happens when points are added and removed as close to the boundary as possible, i.e., when $M \vee M_w = 1$ where the notch only occurs in the higher end of the spectrum. As this is still not ideal, another method for reducing artefacts that less affects the frequency content of the system should be devised, if possible. 

The modal analysis in Figure \ref{fig:modalAnalysis} shows that method generates $ N$ rather than $N - 1$ modes as set by r3. However, the output does contain the correct number of modes as shown in Figure \ref{fig:spectrogram} due to the highest mode not being excited. This is a result of the rigid connection imposed on the inner boundaries, forcing them to have the same displacement and act as one point. As mentioned in the results, when the system is excited at a non-integer $\Nfrac$, the highest mode is excited due to the fact that the inner boundaries can have different displacements in that case, but this is not an issue.

The latter part of r3, however, is not satisfied. The modes deviate from integer multiples of $f_0$, moreso for higher modes. Other interpolation techniques could be investigated to improve the behaviour and decrease this deviation.

Finally, the method only adds a few extra calculations for the inner boundaries so r4 is also easily satisfied. 
\\
% The fact that identical behaviour is observed at integer $N$ proves that at integer $N$ the method indeed reduces to the normal case confirming Section \ref{sec:systSetup}. If the system is excited when $N$ is not an integer, the highest mode will also be excited.

Figure \ref{fig:spectrogramDamp} shows that the displacement correction removes higher frequency modes, but perhaps too much. As the displacement correction acts as a filtering operation, higher speeds of parameter variation will possibly cause the highest mode to not be filtered out `in time'. The values of $\omega_0$ and $\sigma_0$ in Eq. \eqref{eq:dispCorrForce} could be made dependent on the rate of change of $c$ to have a higher effect when $c$ is increased faster.

Although the results bring forward some drawbacks of the method, such as frequency deviations, and low-passing effects, most of these happen affect the higher frequencies of the output. First of all, human frequency sensitivity becomes very limited above 3000 Hz \cite{Zwicker1990} making high-frequency deviations much less important perceptually. Secondly, the physical systems one usually tries to model contain high-frequency losses, causing higher modes to usually not have very high amplitudes. Finally, $\Nfrac$ is usually much bigger in the systems that one tries to model, which decreases the deviations of higher frequencies from perfect harmonics. 

% the analysis has been done on an extreme case: a lossless system with a small number of points. The issues presented, such as frequency deviations happen in higher frequencies. Physical processes usually lose the highest frequencies first so modes with higher frequencies usually do not have high amplitudes. This makes the frequency deviations of higher frequencies and the low-passing effect of the displacement correction much less apparent. 
