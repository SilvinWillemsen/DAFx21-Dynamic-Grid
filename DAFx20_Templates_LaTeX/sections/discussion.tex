\section{Discussion}
To decide whether the proposed method works satisfactory, the results presented in the previous section are compared to the method requirements listed in Section \eqref{sec:methodReq}. 

It can be argued that the frequency deviations of $f_1$ from $f_0$ are sufficiently small to say that the requirement 1 is satisfied. As for requirement 2, a fractional $N$ has been introduced and smooth transitions are indeed observed from Figure \ref{fig:spectrogram}, in the case when $c$ is decreased and $N$ is increased. When $c$ is increased instead, transitions are not smooth as artefacts are generated when grid points are removed. This problem is solved by the displacement correction presented in Section \ref{sec:dispCorr}. However, the filtering effect that the displacement correction has on the system (mentioned in Section \ref{sec:dispCorrRes}) is not ideal. If possible, another method for reducing artefacts that less affects the frequency content of the system should be devised. 

The modal analysis in Figure \ref{fig:modalAnalysis} shows that method generates $\lfloor N\rfloor$ rather than $\lfloor N\rfloor - 1$ modes as set by requirement 3. However, the output does contain the correct number of modes as shown in Figure \ref{fig:spectrogram} due to the highest mode not being excited. This is probably a result of the rigid connection imposed on the inner boundaries, forcing them to have the same displacement. As mentioned in the results, when the system is excited at a non-integer $N$, the highest mode is excited due to the fact that the inner boundaries can have different displacements in that case.

The latter part of requirement 3 is not satisfied. The modes deviate from integer multiples of $f_0$, moreso for higher modes. Luckily, deviations decrease as $N$ increases and most implementations contain much more points than the one tested on here. 

The method only adds a few extra calculations for the inner boundaries so requirement 4 is also easily satisfied. 

% The fact that identical behaviour is observed at integer $N$ proves that at integer $N$ the method indeed reduces to the normal case confirming Section \ref{sec:systSetup}. If the system is excited when $N$ is not an integer, the highest mode will also be excited.

The method has been tested The issues with the proposed method happen in higher frequencies. Physical processes usually lose the highest frequencies first so modes with higher frequencies usually do not have high amplitudes.

Regarding the displacement correction the best consistent low-passing effect regardless of the number of points, it is wise to add and remove points as close to the boundary as possible

As can be seen from the figure, the lower harmonic partials of both systems line up almost exactly before the partials interpolated start to deviate towards the lower end of the spectrum.


Applications could be non-linear systems where parameters are modulated based on the state of the system. 

The proposed method might not provide the exact solution to the problem, but does circumvent the need for upsampling and higher orders of computations necessary to approximate this solution. Even though interpolation needs to happen, the drawbacks of full-grid interpolation can be avoided by not `listening' to the location where points are added but rather closer to the boundary. If one wants to listen to the center, the location where points are added or removed can easily be changed.


frequency domain, the locations of the partials comparing the discrete 1D wave with $N = 30.5$ and $f_\text{s} = 44100$ (interpolation needs to happen) with $N = 61$ and $f$ 

If the amount that a parameter changes within a small enough period of time (give values here, hopefully referring back to the results) the fact that points are added at a specific location (rather than distributed over the grid) will not matter as... 

The 


Note: Reference in Stefan's book about frequency sensitivity > 3000 Hz: E. Zwicker and H. Fastl. Psychoacoustics: Facts and Models. Springer-Verlag, Berlin-Heidelberg,
Germany, 1990.
